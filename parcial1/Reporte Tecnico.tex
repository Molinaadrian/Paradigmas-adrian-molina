 
\documentclass[10pt]{article}         %% What type of document you're writing.
\usepackage{graphicx}
\usepackage{hyperref}
\usepackage[dvipsnames]{xcolor}

%%%%% Preamble

%% Packages to use

\usepackage{amsmath,amsfonts,amssymb}   %% AMS mathematics macros

%% Title Information.

\title{Reporte tecnico}
\author{Jesus Adrian Molina Hernandez}
%% \date{2 July 2004}           %% By default, LaTeX uses the current date

%%%%% The Document

\begin{document}

\maketitle


\section{Introduccion}

En este artículo aplicaremos las mismas para resolver tareas de clasificación en el plano. 1.
Una RNA (Red Neuronal Artificial) es un modelo matemático inspirado en el comportamiento
biológico de las neuronas y en la estructura del cerebro. Esta también puede ser vista como un
sistema inteligente que lleva a cabo tareas de manera distinta a como lo hacen las
computadoras actuales. Si bien estas ´ultimas son muy rápidas en el procesamiento de la
información, existen tareas muy complejas, como el reconocimiento y clasificación de
patrones, que demandan demasiado tiempo y esfuerzo a un en las computadoras más
potentes de la actualidad, pero que el cerebro humano es más apto para resolverlas, muchas
veces sin aparente esfuerzo (considere el lector como ejemplo el reconocimiento de un rostro
familiar entre una multitud de otros rostros). El cerebro puede considerarse un sistema
altamente complejo. Su unidad básica, la neurona, esta masivamente distribuida con
conexiones entre ellas (se calcula que hay aproximadamente 10 billones de neuronas en la
corteza cerebral y 60 trillones de conexiones neuronales).
Si bien hay distintos tipos de neuronas biológicas, en la figura 1 se muestra un esquema
simplificado de un tipo particular que es muy común. Vemos que la misma está compuesta
por:
El cuerpo central, llamado soma, que contiene el núcleo celular Una prolongación del soma, el
axón Una ramificación terminal, las dendritas Una zona de conexión entre una neurona y otra,
conocida como sinapsis

\section{Desarrollo}

Las redes neuronales artificiales están basadas en el funcionamiento de las redes de neuronas
biológicas. Las neuronas que todos tenemos en nuestro cerebro están compuestas de
dendritas, el soma y el axón:
\section{Conclusion}


\end{document}